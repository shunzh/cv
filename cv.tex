%%%%%%%%%%%%%%%%%%%%%%%%%%%%%%%%%%%%%%%%%%%%%%%%%%%%%%%%%%%%%%%%%%%%%%%%
%%%%%%%%%%%%%%%%%%%%%% Simple LaTeX CV Template %%%%%%%%%%%%%%%%%%%%%%%%
%%%%%%%%%%%%%%%%%%%%%%%%%%%%%%%%%%%%%%%%%%%%%%%%%%%%%%%%%%%%%%%%%%%%%%%%

%%%%%%%%%%%%%%%%%%%%%%%%%%%%%%%%%%%%%%%%%%%%%%%%%%%%%%%%%%%%%%%%%%%%%%%%
%% NOTE: If you find that it says                                     %%
%%                                                                    %%
%%                           1 of ??                                  %%
%%                                                                    %%
%% at the bottom of your first page, this means that the AUX file     %%
%% was not available when you ran LaTeX on this source. Simply RERUN  %%
%% LaTeX to get the ``??'' replaced with the number of the last page  %%
%% of the document. The AUX file will be generated on the first run   %%
%% of LaTeX and used on the second run to fill in all of the          %%
%% references.                                                        %%
%%%%%%%%%%%%%%%%%%%%%%%%%%%%%%%%%%%%%%%%%%%%%%%%%%%%%%%%%%%%%%%%%%%%%%%%

%%%%%%%%%%%%%%%%%%%%%%%%%%%% Document Setup %%%%%%%%%%%%%%%%%%%%%%%%%%%%

% Don't like 10pt? Try 11pt or 12pt
\documentclass[10pt]{article}

% The automated optical recognition software used to digitize resume
% information works best with fonts that do not have serifs. This
% command uses a sans serif font throughout. Uncomment both lines (or at
% least the second) to restore a Roman font (i.e., a font with serifs).
%\usepackage{times}
%\renewcommand{\familydefault}{\sfdefault}

% This is a helpful package that puts math inside length specifications
\usepackage{calc}
\usepackage{comment}

% Simpler bibsection for CV sections
% (thanks to natbib for inspiration)
\makeatletter
\newlength{\bibhang}
\setlength{\bibhang}{1em} %1em}
\newlength{\bibsep}
 {\@listi \global\bibsep\itemsep \global\advance\bibsep by\parsep}
\newenvironment{bibsection}%
        {\begin{enumerate}{}{%
%        {\begin{list}{}{%
       \setlength{\leftmargin}{\bibhang}%
       \setlength{\itemindent}{-\leftmargin}%
       \setlength{\itemsep}{\bibsep}%
       \setlength{\parsep}{\z@}%
        \setlength{\partopsep}{0pt}%
        \setlength{\topsep}{0pt}}}
        {\end{enumerate}\vspace{-.6\baselineskip}}
%        {\end{list}\vspace{-.6\baselineskip}}
\makeatother

% Layout: Puts the section titles on left side of page
\reversemarginpar

%
%         PAPER SIZE, PAGE NUMBER, AND DOCUMENT LAYOUT NOTES:
%
% The next \usepackage line changes the layout for CV style section
% headings as marginal notes. It also sets up the paper size as either
% letter or A4. By default, letter was used. If A4 paper is desired,
% comment out the letterpaper lines and uncomment the a4paper lines.
%
% As you can see, the margin widths and section title widths can be
% easily adjusted.
%
% ALSO: Notice that the includefoot option can be commented OUT in order
% to put the PAGE NUMBER *IN* the bottom margin. This will make the
% effective text area larger.
%
% IF YOU WISH TO REMOVE THE ``of LASTPAGE'' next to each page number,
% see the note about the +LP and -LP lines below. Comment out the +LP
% and uncomment the -LP.
%
% IF YOU WISH TO REMOVE PAGE NUMBERS, be sure that the includefoot line
% is uncommented and ALSO uncomment the \pagestyle{empty} a few lines
% below.
%

%% Use these lines for letter-sized paper
\usepackage[paper=letterpaper,
            %includefoot, % Uncomment to put page number above margin
            marginparwidth=1.2in,     % Length of section titles
            marginparsep=.05in,       % Space between titles and text
            margin=1in,               % 1 inch margins
            includemp]{geometry}

%% Use these lines for A4-sized paper
%\usepackage[paper=a4paper,
%            %includefoot, % Uncomment to put page number above margin
%            marginparwidth=30.5mm,    % Length of section titles
%            marginparsep=1.5mm,       % Space between titles and text
%            margin=25mm,              % 25mm margins
%            includemp]{geometry}

%% More layout: Get rid of indenting throughout entire document
\setlength{\parindent}{0in}

\usepackage[shortlabels]{enumitem}

%% Reference the last page in the page number
%
% NOTE: comment the +LP line and uncomment the -LP line to have page
%       numbers without the ``of ##'' last page reference)
%
% NOTE: uncomment the \pagestyle{empty} line to get rid of all page
%       numbers (make sure includefoot is commented out above)
%
\usepackage{fancyhdr,lastpage}
\pagestyle{fancy}
%\pagestyle{empty}      % Uncomment this to get rid of page numbers
\fancyhf{}\renewcommand{\headrulewidth}{0pt}
\fancyfootoffset{\marginparsep+\marginparwidth}
\newlength{\footpageshift}
\setlength{\footpageshift}
          {0.5\textwidth+0.5\marginparsep+0.5\marginparwidth-2in}
\lfoot{\hspace{\footpageshift}%
       \parbox{4in}{\, \hfill %
                    \arabic{page} of \protect\pageref*{LastPage} % +LP
%                    \arabic{page}                               % -LP
                    \hfill \,}}

% Finally, give us PDF bookmarks
\usepackage{color,hyperref}
\definecolor{darkblue}{rgb}{0.0,0.0,0.3}
\hypersetup{colorlinks,breaklinks,
            linkcolor=darkblue,urlcolor=darkblue,
            anchorcolor=darkblue,citecolor=darkblue}

%%%%%%%%%%%%%%%%%%%%%%%% End Document Setup %%%%%%%%%%%%%%%%%%%%%%%%%%%%


%%%%%%%%%%%%%%%%%%%%%%%%%%% Helper Commands %%%%%%%%%%%%%%%%%%%%%%%%%%%%

% The title (name) with a horizontal rule under it
% (optional argument typesets an object right-justified across from name
%  as well)
%
% Usage: \makeheading{name}
%        OR
%        \makeheading[right_object]{name}
%
% Place at top of document. It should be the first thing.
% If ``right_object'' is provided in the square-braced optional
% argument, it will be right justified on the same line as ``name'' at
% the top of the CV. For example:
%
%       \makeheading[\emph{Curriculum vitae}]{Your Name}
%
% will put an emphasized ``Curriculum vitae'' at the top of the document
% as a title. Likewise, a picture could be included:
%
%   \makeheading[\includegraphics[height=1.5in]{my_picutre}]{Your Name}
%
% the picture will be flush right across from the name.
\newcommand{\makeheading}[2][]%
        {\hspace*{-\marginparsep minus \marginparwidth}%
         \begin{minipage}[t]{\textwidth+\marginparwidth+\marginparsep}%
             {\large \bfseries #2 \hfill #1}\\[-0.15\baselineskip]%
                 \rule{\columnwidth}{1pt}%
         \end{minipage}}

% The section headings
%
% Usage: \section{section name}
\renewcommand{\section}[1]{\pagebreak[3]%
    \hyphenpenalty=10000%
    \vspace{1.3\baselineskip}%
    \phantomsection\addcontentsline{toc}{section}{#1}%
    \noindent\llap{\scshape\smash{\parbox[t]{\marginparwidth}{\raggedright #1}}}%
    \vspace{-\baselineskip}\par}

% An itemize-style list with lots of space between items
\newenvironment{outerlist}[1][\enskip\textbullet]%
        {\begin{itemize}[#1,leftmargin=*]}{\end{itemize}%
         \vspace{-.6\baselineskip}}

% An environment IDENTICAL to outerlist that has better pre-list spacing
% when used as the first thing in a \section
\newenvironment{lonelist}[1][\enskip\textbullet]%
        {\begin{list}{#1}{%
        \setlength{\partopsep}{0pt}%
        \setlength{\topsep}{0pt}}}
        {\end{list}\vspace{-.6\baselineskip}}

% An itemize-style list with little space between items
\newenvironment{innerlist}[1][\enskip\textbullet]%
        {\begin{itemize}[#1,leftmargin=*,parsep=0pt,itemsep=0pt,topsep=0pt,partopsep=0pt]}
        {\end{itemize}}

% An environment IDENTICAL to innerlist that has better pre-list spacing
% when used as the first thing in a \section
\newenvironment{loneinnerlist}[1][\enskip\textbullet]%
        {\begin{itemize}[#1,leftmargin=*,parsep=0pt,itemsep=0pt,topsep=0pt,partopsep=0pt]}
        {\end{itemize}\vspace{-.6\baselineskip}}

% To add some paragraph space between lines.
% This also tells LaTeX to preferably break a page on one of these gaps
% if there is a needed pagebreak nearby.
\newcommand{\blankline}{\quad\pagebreak[3]}
\newcommand{\halfblankline}{\quad\vspace{-0.5\baselineskip}\pagebreak[3]}

% Uses hyperref to link DOI
\newcommand\doilink[1]{\href{http://dx.doi.org/#1}{#1}}
\newcommand\doi[1]{doi:\doilink{#1}}

% For \url{SOME_URL}, links SOME_URL to the url SOME_URL
\providecommand*\url[1]{\href{#1}{#1}}
% Same as above, but pretty-prints SOME_URL in teletype fixed-width font
\renewcommand*\url[1]{\href{#1}{\texttt{#1}}}

% For \email{ADDRESS}, links ADDRESS to the url mailto:ADDRESS
\providecommand*\email[1]{\href{mailto:#1}{#1}}
% Same as above, but pretty-prints ADDRESS in teletype fixed-width font
%\renewcommand*\email[1]{\href{mailto:#1}{\texttt{#1}}}

%\providecommand\BibTeX{{\rm B\kern-.05em{\sc i\kern-.025em b}\kern-.08em
%    T\kern-.1667em\lower.7ex\hbox{E}\kern-.125emX}}
%\providecommand\BibTeX{{\rm B\kern-.05em{\sc i\kern-.025em b}\kern-.08em
%    \TeX}}
\providecommand\BibTeX{{B\kern-.05em{\sc i\kern-.025em b}\kern-.08em
    \TeX}}
\providecommand\Matlab{\textsc{Matlab}}

%%%%%%%%%%%%%%%%%%%%%%%% End Helper Commands %%%%%%%%%%%%%%%%%%%%%%%%%%%

%%%%%%%%%%%%%%%%%%%%%%%%% Begin CV Document %%%%%%%%%%%%%%%%%%%%%%%%%%%%

\begin{document}
\makeheading{Shun Zhang}

\section{Contact Information}

% NOTE: Mind where the & separators and \\ breaks are in the following
%       table.
%
% ALSO: \rcollength is the width of the right column of the table
%       (adjust it to your liking; default is 1.85in).
%
\newlength{\rcollength}\setlength{\rcollength}{1.4in}%
%
\begin{tabular}[t]{@{}p{\textwidth-\rcollength}p{\rcollength}}
\href{http://cs.utexas.edu/}%
     {Department of Computer Science} & \\
\href{http://www.utexas.edu/}{The University of Texas at Austin} \\
1 University Station A8000   & 512-574-3694 \\
Austin, TX  78712     & \email{jensen.zhang@utexas.edu}\\
\end{tabular}

\section{Research Interests}

Reinforcement learning, robotics, theoretical machine learning, human cognition.

\section{Education}

\href{http://www.utexas.edu}{\textbf{University of Texas at Austin}},
Austin, TX
\begin{outerlist}

\item[] Integrated B.S./M.S. Program,
        \href{http://cs.utexas.edu/}
             {Computer Science},
             Jan. 2012 - May. 2015 (Expected)
        \begin{innerlist}
        \item Major G.P.A. 3.8. Overall G.P.A. 3.55.
        \item Master Thesis with Prof. Peter Stone.
        \end{innerlist}
\end{outerlist}
\vspace{.1in}
\href{http://iao.nuaa.edu.cn/}{\textbf{Nanjing University of Aeronautics and
Astronautics}},
Nanjing, China
\begin{outerlist}
\item[] Undergraduate program,
             Computer Science and Technology,
             Sep. 2009 - Dec. 2011
        \begin{innerlist}
        \item G.P.A. 88/100.
        \item Transferred to University of Texas at Austin in Jan. 2012.
        \end{innerlist}

\end{outerlist}

\section{Research Experience}

%\textbf{Representation Learning in Reinforcement Learning} \hfill {Fall 2014-Spring 2015}
%\label{lk:masterThesis}
%\begin{innerlist}
%\item[] Department of Computer Science\\
%        University of Texas at Austin
%\vspace{.05in}
%\item Supervisor: Prof. \href{http://www.cs.utexas.edu/~pstone/}{Peter Stone}.
%\item Research question: {\em Reinforcement Learning needs abstraction for
%large-scale problems. Examples are feature extraction and hierarchical learning.
%Can the learning agent learn such abstraction on the fly?}
%\item In progress for Master Thesis.
%\end{innerlist}
%\vspace{.1in}

\textbf{Modular Reinforcement Learning} \hfill {Fall 2014}
\begin{innerlist}
\item[] Department of Computer Science and Center for Perceptual Systems\\
        University of Texas at Austin
\vspace{.05in}
\item Supervisor: Prof. \href{http://www.cs.utexas.edu/~dana/}{Dana Ballard} and
Prof. \href{http://www.utexas.edu/cola/centers/cps/faculty/mmh739}{Mary Hayhoe}.
\item Research question: {\em Assume human already has Markov Decision Processes
(MDP) trained for preliminary tasks, how would these MDPs contribute to the complicated
behavior?}
\item Using Inverse Reinforcement Learning to interpret human's behavior,
assuming that it is a combination of the MDPs for preliminary tasks.
\end{innerlist}
\vspace{.1in}

\textbf{Determining Placements of Influencing Agents in a Flock} \hfill {Fall 2014}
\begin{innerlist}
\item[] Department of Computer Science\\
        University of Texas at Austin
\vspace{.05in}
\item Supervisor: Prof. \href{http://www.cs.utexas.edu/~pstone/}{Peter Stone}.
\item Research question: {\em Where should influencing agents be located within
a flock to maximize their influence on the flock?}
\item Using MASON simulator to evaluate different placements, including border,
grid, and graph-based placements.
\item Paper in preparation: Determining Placements of Influencing Agents in a
Flock. Katie Genter, Shun Zhang and Peter Stone.
\end{innerlist}
\vspace{.1in}

\textbf{Analysis of Reinforcement Learning Convergence Time using Mixing Time}
\hfill {Fall 2014}
\begin{innerlist}
\item[] Department of Mathematics\\
        University of Texas at Austin
\vspace{.05in}
\item Supervisor: Dr. \href{http://joeneeman.com}{Joe Neeman}.
\item Research question: {\em Can we prove a theoratical bound of the
convergence time for popular RL algorithms?}
\item Markov Chain and Mixing Time course project.
\end{innerlist}
\vspace{.1in}

\textbf{Action Selection in Robotic Motion Learning} \hfill {Fall 2013}
\begin{innerlist}
\item[] Department of Computer Science\\
        University of Texas at Austin
\vspace{.05in}
\item Supervisor: Prof. \href{http://www.cs.utexas.edu/~pstone/}{Peter Stone}.
\item Research question: {\em Instead of uniformly randomly selecting actions to
try, can a robot explicitly select actions to explore its belief state space?}
\item Implementing ASAMI (a model-learning algorithm) on Nao robot using
bandit-based exploration.
\item Autonomous Robots course project. Achieved in Undergraduate Research
Journal in University of Texas at Austin, 2014.
\end{innerlist}
\vspace{.1in}

\textbf{Structured Exploration for Relational Reinforcement Learning} \hfill
{Spring 2013}
\begin{innerlist}
\item[] Department of Computer Science\\
        University of Texas at Austin
\vspace{.05in}
\item Supervisor: Prof. \href{http://www.cs.utexas.edu/~pstone/}{Peter Stone}.
\item Research question: {\em Can we improve the exploration efficiency of the
Relational Reinforcement Learning algorithm?}
\item Applying the exploration machanism in Rmax-Q to Relational Reinforcement
Learning to improve the latter's sample efficiency.
\item Reinforcement Learning course project.
\end{innerlist}
\vspace{.1in}

\textbf{Semi-Autonomous Intersection Management} \hfill {Summer, Fall 2012}
\begin{innerlist}
\item[] Department of Computer Science\\
        University of Texas at Austin
\vspace{.05in}
\item Supervisor: Prof. \href{http://www.cs.utexas.edu/~pstone/}{Peter Stone}
and Prof. \href{http://cse.unist.ac.kr/~chiu}{Tsz-Chiu Au}.
\item Research question: {\em Can we find a policy better than traffic signals,
if human-driven, semi-autonomous and fully-autonomous vehicles are sharing the
road?}
\item Designing and evaluating a policy that is competent with all three types
of vehicles, and performs better than traffic signals.
\item Related publication: Semi-Autonomous Intersection Management (Extended
Abstract). Tsz-Chiu Au, Shun Zhang, and Peter Stone. Autonomous Agents and
Multiagent Systems (AAMAS), 2014.
\end{innerlist}

\section{Publications}
\begin{innerlist}
\item Tsz-Chiu Au, {\bf Shun Zhang}, and Peter Stone. Semi-Autonomous
Intersection Management (Extended Abstract).  Autonomous Agents and Multiagent
Systems (AAMAS), 2014.
\end{innerlist}

\section{Papers in Preparation}
\begin{innerlist}
\item Katie Genter, {\bf Shun Zhang}, and Peter Stone. Determining Placements of
Influencing Agents in a Flock. 
\end{innerlist}

\section{Presentation}
\begin{innerlist}
\item Intersection Management with Constraint-Based Reservation Systems. 
Autonomous Robots and Multirobot Systems (ARMS), 2014.
\end{innerlist}

\section{Conference Attendance}
\begin{innerlist}
\item Autonomous Agents and Multiagent Systems (AAMAS), Paris, 2014.
\end{innerlist}

\section{Awards}
Student Awards --- University of Texas at Austin
\begin{innerlist}
\item Louis E. Rosier Memorial Endowment Scholarship. \hfill 2013-2014
\end{innerlist}
\vspace{.1in}

Student Awards --- Nanjing University of Aeronautics and Astronautics
\begin{innerlist}
\item Department Scholarships. \hfill 2009-2011
\end{innerlist}

\halfblankline

\section{Courses Taken}
Graduate Level
\begin{innerlist}
\item Large Scale Optimization (EE 381V)
\item Markov Chain and Mixing Time (M 394C)
\item Machine Learning (CS 391L)
\item Autonomous Robots (CS 393R)
\item Randomized Algorithms (CS 388R)
\item Reinforcement Learning (CS 394R)
\end{innerlist}
Undergraduate Level
\begin{innerlist}
\item Artificial Intelligence (CS 343)
\item Principles of Computer Systems (CS 439)
\item Automata Theory (CS 341)
\item Information Retrieval (CS 371R)
\item Programming Languages (CS 345)
\item etc.
\end{innerlist}

\section{Teaching Experience}

\textbf{Undergraduate Teaching Assistant (Proctor)} \hfill {Fall 2013, Spring 2014}
\begin{innerlist}
\item[] CS 301K Foundations of Logical Thought\\
        with Dr. Jacob Schrum\\
        Department of Computer Science,\\
        University of Texas at Austin
\end{innerlist}

\section{Industrial Experience}

\textbf{SDE Intern at Amazon} \hfill {Summer 2014}
\begin{innerlist}
\item[] Seattle, WA
\end{innerlist}
\vspace{.1in}

\textbf{SDE Intern at Semantic Designs} \hfill {Summer 2013}
\begin{innerlist}
\item[] Austin, TX
\end{innerlist}

\section{Languages}

\begin{innerlist}
\item Natural languages: Mandarin Chinese (native), English (fluent), Japanese
(preliminary).
\item Programming languages: Proficient in programming in Python, Octave/Matlab,
Java, C/C++; Familiar with Lisp, Oracle SQL, \LaTeX, Web Development Languages
(HTML, JavaScript, PHP), Perl, Scala.
\end{innerlist}




\end{document}

%%%%%%%%%%%%%%%%%%%%%%%%%% End CV Document %%%%%%%%%%%%%%%%%%%%%%%%%%%%%

%----------------------------------------------------------------------%
% The following is copyright and licensing information for
% redistribution of this LaTeX source code; it also includes a liability
% statement. If this source code is not being redistributed to others,
% it may be omitted. It has no effect on the function of the above code.
%----------------------------------------------------------------------%
% Copyright (c) 2007, 2008, 2009, 2010, 2011 by Theodore P. Pavlic
%
% Unless otherwise expressly stated, this work is licensed under the
% Creative Commons Attribution-Noncommercial 3.0 United States License. To
% view a copy of this license, visit
% http://creativecommons.org/licenses/by-nc/3.0/us/ or send a letter to
% Creative Commons, 171 Second Street, Suite 300, San Francisco,
% California, 94105, USA.
%
% THE SOFTWARE IS PROVIDED "AS IS", WITHOUT WARRANTY OF ANY KIND, EXPRESS
% OR IMPLIED, INCLUDING BUT NOT LIMITED TO THE WARRANTIES OF
% MERCHANTABILITY, FITNESS FOR A PARTICULAR PURPOSE AND NONINFRINGEMENT.
% IN NO EVENT SHALL THE AUTHORS OR COPYRIGHT HOLDERS BE LIABLE FOR ANY
% CLAIM, DAMAGES OR OTHER LIABILITY, WHETHER IN AN ACTION OF CONTRACT,
% TORT OR OTHERWISE, ARISING FROM, OUT OF OR IN CONNECTION WITH THE
% SOFTWARE OR THE USE OR OTHER DEALINGS IN THE SOFTWARE.
%----------------------------------------------------------------------%
